%%%% Generic manuscript mode
\documentclass[ manuscript,screen, nonacm]{acmart}

\usepackage{amsmath}
\usepackage{upgreek}
\usepackage{mathtools}
\usepackage{mathpartir}
\usepackage{syntax}
\usepackage{amsthm}
\usepackage{hyperref}
% Grammar
\newcommand{\arr}[2]{#1 \rightarrow #2}
\newcommand{\arrc}[2]{\mathtt{arr}\left(#1,#2\right)}

\newcommand{\unit}{\mathbf{1}}
\newcommand{\unitc}{\mathtt{unit}}

\newcommand{\app}[2]{#1\left(#2\right)}
\newcommand{\appc}[2]{\mathtt{app}\left(#1,#2\right)}

\newcommand{\lam}[3]{\lambda\left(#1:#2\right).~#3}
\newcommand{\lamc}[3]{\mathtt{lam}\{#2\}\left(#1.#3\right)}

\newcommand{\trivc}{\mathtt{triv}}
\newcommand{\triv}{\langle\rangle}

\newcommand{\fixp}[3]{\mathtt{fix}\left(#1:#2\right).~#3}
\newcommand{\fixpc}[3]{\mathtt{fix}\{#2\}\left(#1.#3\right)}

\newcommand{\tick}[2]{\mathtt{tick}~#1~\mathtt{in}~#2}
\newcommand{\tickc}[2]{\mathtt{tick}\{#1\}\left(#2\right)}

% Types
\newcommand{\isval}[1]{#1\,\mathrm{val}}

\newcommand{\costof}[2]{#1\left(\mathtt{#2}\right)}
\newcommand{\eval}[3]{ #1 \Downarrow^{#2} #3}
\newcommand{\evalenv}[5]{#1 \vdash_{#2} #3 \Downarrow^{#4}#5}
\newcommand{\evalwatermark}[6]{#1 {}_{#2}\!\vdash^{#3}_{#4} #5 \Downarrow #6 }
\newcommand{\evalmono}[6]{#1 \vdash_{#2} #3 \Downarrow #4 \mid \left(#5,#6\right)}

\newcommand{\sharingsymbol}{\curlyvee} % replace this by the correct symbol later
\newcommand{\sharing}[3]{#1\curlyvee#2\mid#3}

\newcommand{\typ}[5]{#1\vdash^{#2}_{#3} #4:#5}
\newcommand{\matchc}[5]{\mathtt{matL}\left(#1,#2,#3.#4.#5\right)}


\usepackage[english]{babel}
 
\usepackage{amsthm}
 
\theoremstyle{definition}
\newtheorem{definition}{Definition}[section]
\newtheorem{remark}{Remark}[section]
\newtheorem*{remark*}{Remark}


\newcommand{\share}[4]{\mathtt{share}~#1~\mathtt{as}~#2,#3~\mathtt{in}~#4}

\usepackage{listings}
\usepackage{color}

\definecolor{dkgreen}{rgb}{0,0.6,0}
\definecolor{gray}{rgb}{0.5,0.5,0.5}
\definecolor{mauve}{rgb}{0.58,0,0.82}

\lstset{frame=tb,
  language=SML,
  aboveskip=3mm,
  belowskip=3mm,
  showstringspaces=false,
  columns=flexible,
  basicstyle={\small\ttfamily},
  numbers=none,
  numberstyle=\tiny\color{gray},
  keywordstyle=\color{blue},
  commentstyle=\color{dkgreen},
  stringstyle=\color{mauve},
  breaklines=true,
  breakatwhitespace=true,
  tabsize=3
}

% Commands for Bob Harper's lecture

\newcommand{\clsfd}{\underline{\mathtt{clsfd}}}
\newcommand{\inharp}[2]{\underline{\mathtt{in}}\left[#1\right](#2)}
\newcommand{\isin}[4]{\underline{\mathtt{isin}}\left[#1\right](#2,#3,#4)}
\newcommand{\new}[2]{\underline{\mathtt{new}}\left[#1\right](#2)}
\newcommand{\ret}[1]{\underline{\mathtt{ret}}\left(#1\right)}
\newcommand{\clsty}{\underline{\mathtt{cls}}}
\newcommand{\cls}[1]{\underline{\mathtt{cls}}\left[#1\right]}
\newcommand{\reveal}[2]{\underline{\mathtt{reveal}}(#1; #2)}
\newcommand{\inrefharp}[2]{\underline{\mathtt{inref}}(#1; #2)}
\newcommand{\isinref}[4]{\underline{\mathtt{isinref}}(#1; #2; #3; #4)}
\newcommand{\mk}{\mathtt{mk}}
\newcommand{\option}{\underline{\mathtt{opt}}}
\newcommand{\bnd}[3]{\underline{\mathtt{bnd}}(#1, #2, #3)}
\newcommand{\cmp}[1]{\underline{\mathtt{comp}(#1)}}

\newcommand{\approxty}{%
  \mathbin{\text{%
    \mathsurround=0pt
    \ooalign{%
      \hidewidth\vphantom{$\div$}\raisebox{.95ex}{\scalebox{1.2}{.}}\hidewidth\cr
      $\sim$\cr
      \hidewidth\vphantom{$\div$}\raisebox{-.05ex}{\scalebox{1.2}{.}}\hidewidth\cr
    }%
  }}%
}

%%
%% \BibTeX command to typeset BibTeX logo in the docs
% \AtBeginDocument{%
%   \providecommand\BibTeX{{%
%     \normalfont B\kern-0.5em{\scshape i\kern-0.25em b}\kern-0.8em\TeX}}}

\begin{document}

\author{Jiawen Liu}
\email{jliu223@buffalo.edu}

\author{J\'er\'emy Thibault}
\email{jeremy.thibault@inria.fr}

\author{Shaowei Zhu}
\email{shaoweiz@cs.princeton.edu}

\title{Lecture Notes on Practical Foundations for Programming Languages}
\subtitle{Part 4: Dynamic Classification}

\maketitle

These are lecture notes taken during Robert Harper's course on foundations of
programming languages at the Oregon Programming Language Summer School 2019.
See also a \textnf{PFPL} supplement on exceptions at \href{http://www.cs.cmu.edu/~rwh/pfpl/supplements/exceptions.pdf}{http://www.cs.cmu.edu/~rwh/pfpl/supplements/exceptions.pdf}.

\section{Motivation}

In this lecture, the goal is to build an exception control mechanism that:
\begin{itemize}
    \item does not rely on any convention about the kind of data that is associated
    with an exception
    \item that ensures integrity and confidentiality of data: no third party should be able
    to intercept an exception intended for a particular handler
\end{itemize}

This is implemented by using dynamically created classes of data, and associating
capabilities to them capabilities representing integrity and confidentiality of the
associated data.

\section{Static Classification}
Dynamic classifications is to oppose to static classification, i.\@e\@., types built in the form of finite sums.
Indeed, in a programming language equipped with static classification, 
a type is computed by solving a type equation: 
\[
\mathcal{U} \cong \big[
\mathtt{num} \hookrightarrow \omega, 
\mathtt{nil} \hookrightarrow \mathbf{1}, 
\mathtt{cons} \hookrightarrow \mathcal{U} \times \mathcal{U}, 
\mathtt{fun} \hookrightarrow \mathcal{U} \rightarrow \mathcal{U}, ...
\big] 
\]
where each summand is a class of values.

Dynamic dispatch can be seen as a way to work with the data typed as above.
When we want to manipulate a value of this data type, we need to \emph{unfold} it and
then pattern-match on the type.
\[
\mathtt{case}(u, \mathtt{num} \hookrightarrow \dots; \dots; \mathtt{cons}
  \hookrightarrow \langle x_1, x_2\rangle)
\]
The case analysis reveals further information about the type of the data that makes downstream manipulations possible.

Datatypes are \emph{abstract} (statically "new": even you write the same implementation twice, they are two different types) recursive \emph{sums} (static).
Hence, this is an issue of using \emph{generative} datatypes vs \emph{concrete} datatypes.

\section{Dynamic Classification}

Now, we want a \emph{dynamically extensible} sum: \(\tau_1 + \tau_2 + \ldots\), generating
new datatypes at runtime.

This idea of dynamically ``new'' connects closely to the notions of information-flow or 
knowledge-flow: a dynamic new class corresponds to an unguessable secret (or perfect
encryption, by \(\alpha\)-conversion), 
% JT: I didn't really understand this part
% SZ: I think he wanted to say that he wants a mechanism where the naming of exceptions does not matter. But I could not find a clear way to express this.
and one could control the flow of information by choosing who knows the secret.

We obtain abstract datatypes, with \emph{induced capabilities} that express integrity and confidentiality, where
\begin{itemize}
    \item the implementer is responsible for the integrity of data;
    \item the client is responsible for confidentiality.
\end{itemize}

In particular, this may be used to implement exceptions: ``shared secrets'' between a raiser
and a particular handler. In this case,
\begin{itemize}
    \item the raiser is the implementer responsible for the integrity of the exception
    \item handlers other than the designated handler cannot decode the secret, thus are unable to handle the exception while being well-typed
\end{itemize}

Let us assume that there already exists an exception control-flow mechanism in our language.
We will not focus on implementing the exception data itself.

\section{Implementation of the exception datatype}

\paragraph{Idea} We are going to associate with each exception  a new class. 
An exception is raised by specifying its class;
and an exception is handled by dispatching on the class.
% \textbf{idea}: associate a class that shows how to encrypt it

\subsection{Statics}

\begin{align*}
    \textrm{\emph{<Type>}}~\tau &::= \cdots \mid \clsfd\\
    \textrm{\emph{<Value>}}~v &::= \cdots \mid \inharp{c}{v} \quad\quad & :\clsfd\\
    \textrm{\emph{<Comp>}}~m &::= \cdots \mid \isin{c}{v}{x.m_1}{m_2} &\approxty \tau\\
    &\phantom{::=} \mid \new{\tau}{c.m}
\end{align*}

Here, 
\(c\) represents a class, 
used as an indexing scheme to distinguish between different exceptions. 
The computation \(\isin{c}{v}{x.m_1}{m_2}\) type checks as \(\tau\) 
if both \(m_1 \approxty \tau\)
and \(m_2 \approxty \tau\), 
and \(v : \clsfd\). 
Its semantics is that it checks 
if \(v\) is in \(c\), and performs \([x/v]m_1\) in that case, or \(m_2\) otherwise.
Finally, \(\new{\tau}{c.m}\) declares a new class \(c\) in \(m\).

We also update our typing judgments to include an environment \(\Sigma\) of class binders associated with types.
\begin{align*}
    \Gamma \vdash_{\Sigma}& v : \tau\\
    \Gamma \vdash_{\Sigma}& m \approxty \tau
\end{align*}
This environment \(\Sigma\) declares the active classes in the current scope.
Note that, in our formalism, \(c\) is a class binder that is \emph{not} a variable.
In particular, the order in \(\Sigma\) does not matter, it is allowed to weaken \(\Sigma\),
and we can perform \emph{injective} renaming, but not substitute (the renaming must be
injective to preserve disequality).

The typing rule for ``new'' follows:
\begin{mathpar}
\inferrule
{
\Gamma \vdash_{\Sigma, a\sim \tau} m \approxty \tau'
}
{
\Gamma \vdash_\Sigma \new{\tau}{a.m} \approxty \tau'
}
\end{mathpar}


\subsection{Dynamics}

We consider the following:
\begin{itemize}
    \item states $\nu\Sigma\{m\}$: classes $\nu$ that could be used in computation $m$;
    \item transitions $\nu \Sigma \{m\} \mapsto \nu \Sigma'\{m'\}$: where the invariant $\Sigma \subseteq \Sigma'$ holds;
    \item final states $\nu \Sigma \{\ret{v}\}$
\end{itemize}

We introduce a static judgment, meaning that a state is well-formed:
\[
  \nu\Sigma\{m\} \textrm{OK} \triangleq \cdot\vdash_{\Sigma} m\approxty \tau~\text{for any}~\tau
  \]

The semantics of new is:
\[
  \nu\Sigma\{\new{\tau}{c.m}\} \mapsto \nu\Sigma,c\sim\tau\{m\}.
\]
It allocates new classes of data, for ``encryption'' purposes.

The semantics of \texttt{in} and \texttt{isin} are:
\[
  \nu\Sigma,c\sim\tau,c'\sim\tau'\{\isin{c}{\inharp{c'}{v'}}{x.m_1}{m_2} \mapsto
  \begin{cases}
  [v'/x]m_1 \text{ if \(c=c'\)}\\
  m_2 \text{ otherwise}
  \end{cases}.
\]
The first case corresponds to the case where one learns the equality, and is able to perform
the computation. In the second case, nothing at all is learnt.

% \paragraph{Scoping}
% JT: I didn't really get this part. Can someone try to explain it in better terms?
% SZ: I can't understand this either. It's something he mentioned but seems to be isolated from the main development
% $m \mapsto_{\Sigma} m'$ where transition is parameterized by $\Sigma$, is scoped.
% JT: I removed the paragraph. It's probably not that important!

\subsection{Class references}
Finally, we introduce the idea of class references.

\begin{align*}
    \tau &::= \cdots \mid \tau\ \clsty \quad\quad &\text{the type of class references}\\
    v &::= \cls{c} &\text{written as } \&c\\
    m &::= \reveal{v}{c.m}
\end{align*}

The additional typing rules are:
\begin{mathpar}
\inferrule{\empty}{\Gamma\vdash_{\Sigma, c~\tau} \&c : \tau\ \clsty}
\and
\inferrule{\Gamma \vdash_\Sigma v:\tau\ \clsty\\ 
\Gamma \vdash_{\Sigma, c\sim \tau} m  \approxty \tau'}{
\Gamma \vdash_{\Sigma}  \reveal{v}{c.m} \approxty \tau'
}
\end{mathpar}

Reveal has the following semantics:
\[
\nu \Sigma, c\sim\tau\{ \reveal{\&c}{c.m}\} \mapsto \nu\Sigma, c\sim \tau \{m\}
\]
This corresponds to guessing by \(\alpha\)-conversion.

We may now define two shortcuts, that behaves like ``in'' and ``isin'', but for
class references:
\begin{align*}
  \inrefharp{v}{v'} &\triangleq \reveal{v}{c.\inharp{c}{v'}} \approxty \clsfd\\
  \isinref{v}{v'}{x.m_1}{m_2} &\triangleq \reveal{v}{c.\isin{c}{v'}{x.m_1}{m_2}}.
  % \isin{\cdot}{\inrefharp{v}{v'}}{x.m_1}{m_2} 
\end{align*}
% and \(\isinref{v}{v'}{x.m_1}{m_2}\).

\subsection{Possession of a class reference}
Possession of a class reference affords \emph{full rights} to create and analyze classified values.
One may associate to a type \(\tau\ \clsty\) the following tuple:
\[
  \tau\ \clsty \approx 
  \left\{
    \begin{array}{lr}
    \mk: &\tau \rightarrow \clsfd \\
    \mk^{-1}:  &\clsfd \rightarrow \tau\ \option
    \end{array}
  \right\}
\]
and to the corresponding value \(\&a\) the pair:
\[
\& a \mapsto 
\left \{
\begin{array}{l}
  \mk \hookrightarrow \lambda[\tau](x : \inharp{a}{x})\\
  \mk^{-1} 
    \hookrightarrow \lambda[\clsfd](x.\isin{a}{x}{y.\ret{1.y}}{\ret{2.<>}})
    %(x.isin[a](x;y.ret(1.y;ret(2.<>))))
\end{array}
\right\}
\]
% JT: what is y here? I'm a bit confused


The following properties hold:
\begin{itemize}
    \item the possessor of \(\mk\) is held responsible for the integrity of the underlying value, and can enforce the value respects an invariant
    % $a \sim P_a$ rep invariant. for example, make sure that the value adheres to its rep invariant
    \item the possessor of \(\mk^{-1}\) can compromise the confidentiality, and confirm
    that the invariant holds
    % — and know that associated invariant holds.
\end{itemize}

% Using this scheme

% \[
% \&a \left\{ 
% \begin{array}{lr}
%    \text{send a party mk}  &  \\
%    \text{send another party} mk^{-1} & 
% \end{array}
% \right\}
% \]

\section{An application in ML}

Let us introduce some concepts from the ML language.

\begin{itemize}
    \item \(\mathtt{\underline{exception}}\ X\ \mathtt{\underline{of}}\ \tau\) declares an exception
    \(X\) carrying values of type \(\tau\)
    \item \(\mathtt{\underline{raise}}\ e\) where \(e : \mathtt{\underline{exn}}\) (that is,
    \(\clsfd\)) raises an exception
    \item \(e\ \mathtt{\underline{handle}}\ X(x)\) declares an handler. However, one might
    as well write \(z\) instead of \(X(x)\), because one doesn't need to have access
    to the name of the exception to handle it.
\end{itemize}

So what does \(\mathtt{\underline{exception}}\ X\ \mathtt{\underline{of}}\ \tau\) actually mean?

\[
  \mathtt{\underline{val}}\ X =
  \left \{
  \begin{array}{lr}
  \mathtt{\underline{raise}}[\tau] & \hookrightarrow \text{\clsfd} \rightarrow \tau\\
  \mathtt{\underline{handle}}[\tau] & \hookrightarrow \tau \rightarrow (\clsfd \rightarrow \tau) \rightarrow \tau
  \end{array}
  \right\}
\]
The handler takes two parameters: the expression to evaluate, and a function to call
if an exception is raised.

In ML, \(\mathtt{\underline{raise}}\) has the following meaning, where \(X : \tau \rightarrow \clsfd\) and \(x : \tau\):
\[
  \mathtt{\underline{raise}}~X(v) \hookrightarrow \bnd{\cmp{X.\mk(v)}}{y.\mathtt{RAISE}(y)}{
  z.\mathtt{RAISE}(z)}
\]
This computes the content of the exception, classify it, and then calls the system
implementation for raising an exception. The second \(\mathtt{RAISE}\) is used if
the inner computation raises an exception itself.

Similarly, \(\underline{\mathtt{handle}}\) has the following meaning:
\[
  m~\mathtt{\underline{handle}}~X(x) \Rightarrow m' \hookrightarrow
    \bnd{\cmp{m}}{x.\ret{x}}{y.\bnd{\cmp{x.\mk^{-1}(y)}}{x.m'}{z.\mathtt{RAISE}(z)}}
\]
If the execution of \(m\) succeeds, the value is immediately returned. Otherwise,
the exception is decoded (\(\mk^{-1}\)), then handled by \(m'\). Again, one need to handle
the case of an inner exception.

Finally, it is now possible to see exceptions as dynamically allocating a new class:
\[
\underline{\mathtt{exception}}~X~\underline{\mathtt{of}}~\tau \hookrightarrow
\new{\tau}{c.\langle \mk\hookrightarrow \dots, \mk^{-1} \hookrightarrow \dots\rangle}
\]

\end{document}
\endinput