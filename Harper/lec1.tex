\documentclass{article}
\usepackage[utf8]{inputenc}
\usepackage[T1]{fontenc}
\usepackage{amsmath,amssymb,lmodern}
\usepackage{amsthm}
\usepackage{minibox}
\usepackage{pgfplots}
\pgfplotsset{width=10cm,compat=1.9}
\usepackage{tikz-cd}
\usepackage{adjustbox}
\usepackage{circuitikz}
\usepackage{hyperref}
\usepackage{parskip}
\usepackage{mathpartir}

\usepackage{listings}
\usepackage{color}

\usepackage[shortlabels]{enumitem}
\usepackage[bb=boondox]{mathalfa}
\usepackage{ifthen}
\usepackage{multicol}
\usepackage{multirow}
\usepackage{hyperref}
\usepackage{mathpartir}
\usepackage{float}

\makeatletter
\def\namedlabel#1#2{\begingroup
   \def\@currentlabel{#2}%
   \label{#1}\endgroup
}
\makeatother
\newcommand{\infnlabel}[2]{#1\namedlabel{#2}{#1}}
\newcommand{\infref}[1]{\textsc{\ref{#1}}}

\definecolor{eclipseBlue}{RGB}{42,0.0,255}
\definecolor{eclipseGreen}{RGB}{63,127,95}

\newcommand{\Circ}[1]{#1\ \texttt{circ}}
\newcommand{\Eval}[2]{#1 \Downarrow #2}
\newcommand{\ttt}{\texttt{tt}}
\newcommand{\ff}{\texttt{ff}}
\newcommand{\fix}[2]{\texttt{fix}(#1 . #2)}
\newcommand{\RNum}[1]{\uppercase\expandafter{\romannumeral #1\relax}}
\newcommand{\nor}[3]{\overline{V}_{\RNum #1}(#2, #3)}
\newcommand{\multival}[2]{\ensuremath{#1 \mapsto^{*} #2}}

\title{Practical Foundations for Programming Languages}
\author{Jason Hu, Zhe Zhou, Ramana Nagasamudram}
\date{June 19. 2019}

\begin{document}
\maketitle

\section{Essentials}
\begin{itemize}
  \item Practical Foundations for Programming Languages. \url{https://cs.cmu.edu/~rwh/pfpl}
  \item Abstract syntax with binding and scope
  \item Inductive definitions "least relation closed with rules"
  \item Safety for partial languages
\end{itemize}

\begin{displayquote}
``What does it mean for a PL to exist?''

``A mathematical object subject to rigorous analysis.''
\end{displayquote}

\begin{enumerate}
\item abstract syntax
\item statics  / formation
\item dynamics / execution $\to$ the truth about what is going on
\end{enumerate}

All should be coherent.

Statics is approximation of dynamics. Statics is a prediction about the dynamics.

\section{Abstract Syntax}

\begin{align*}
  e ::= \ x\ |\ \ttt\ |\ \ff\ |\ \nor 1{e_1}{e_2}\ |\ \nor 2{e_1}{e_2}\ |\
          \fix{x}{e}
\end{align*}

\section{Statics}

We write $\Circ{e}$ to denote that $e$ is a well-formed circuit.

\begin{mathpar}
  \inferrule*
  { }{\Circ{\ttt}}

  \inferrule*
  { }{\Circ{\ff}}

  \inferrule*
  {\Circ{e_1} \\ \Circ{e_2}}
  {\Circ{\nor{*}{e_1}{e_2}}}

  \inferrule*
  { } {\Circ{x} \vdash \Circ{x}}

  \inferrule*
  {\Circ{x} \vdash \Circ{e}}
  {\Circ{\fix x e}}
\end{mathpar}

A variable is a placeholder, not a direct participant of computation. It stands for a circuit to be determined.

Consequence relations (entailment, ..) specify the meanings of variables.

\section{Dynamics}

"$\Eval{e}{v}$" denotes that $e$ evaluates to $v$.

\begin{mathpar}
  \inferrule*
  { }{\Eval{\ttt}{\ttt}}

  \inferrule*
  { }{\Eval{\ff}{\ff}}

  \inferrule*
  {\Eval{e_1}{\ttt}}
  {\Eval{\nor{1}{e_1}{e_2}}{\ff}}

  \inferrule*
  {\Eval{e_1}{\ff} \\ \Eval{e_2}{\ttt}}
  {\Eval{\nor 1{e_1}{e_2}}{\ff}}

  \inferrule*
  {\Eval{e_1}{\ff} \\ \Eval{e_2}{\ff}}
  {\Eval{\nor 1{e_1}{e_2}}{\ttt}}

  \inferrule*
  {\Eval{e_2}{\ttt}}
  {\Eval{\nor 2{e_1}{e_2}}{\ff}}

  \inferrule*
  {\Eval{e_1}{\ttt} \\ \Eval{e_2}{\ff}}
  {\Eval{\nor 2{e_1}{e_2}}{\ff}}

  \inferrule*
  {\Eval{e_1}{\ff} \\ \Eval{e_2}{\ff}}
  {\Eval{\nor 2{e_1}{e_2}}{\ttt}}

  \inferrule*
  {\Eval{[\fix x e/x]e}{v}}
  {\Eval{\fix x e}{v}}
\end{mathpar}

Fact: If $\emptyset \vdash \Circ{e}$ then there exists $v$, $\Eval{e}{v}$.
\begin{proof}
 Induction on statics.
\end{proof}

\section{Structural Properties of Entailment}

Structural properties characterizes behavior of variables.

\begin{mathpar}
 \inferrule*[right=Reflexivity]
  { }
  {\Circ{x} \vdash \Circ{x}}

  \inferrule*[right=Substitution]
  {\Circ{e_1} \\ \Circ{x} \vdash \Circ{x}}
  {\Circ{[e_1/x]e_2}}

  \inferrule*[right=Weakening]
  {\Circ{e} \\ x \notin e}
  {\Circ{x} \vdash \Circ{e}}

  \inferrule*[right=Exchange]
  {\Circ{x},\Circ{y} \vdash \Circ{e}}
  {\Circ{y},\Circ{x} \vdash \Circ{e}}

  \inferrule*[right=Contraction]
  {\Circ{x},\Circ{y} \vdash \Circ{e}}
  {\Circ{z} \vdash \Circ{[z,z/x,y]e}}
\end{mathpar}

Structural properties express limiting conditions on the language. For example, type is a kind of restrictions.

\hyperlink{Exercises}{Exercises: BDDS}.

\begin{mathpar}
  X \stackrel{?}{=} \overline{V}_{I}(R, \overline{V}_{II}(X, S))
\end{mathpar}

How to deal with recursion?
\begin{align*}
  \begin{cases}
    X = \nor 1 R Y \\
    Y = \nor 1 X S
  \end{cases}
\end{align*}

Expanding $X$ gives
\begin{align*}
    X = \nor 1 R{\nor 2 X S}
\end{align*}

Can we take this self reference formula and solve for X? We assume X = $e_X$, and is equal to $f(X)$. Seek a fixpoint for f. which is a f make the things in the left hand and right hand the same.

That is the "Y combinator" a.k.a. "fixed point combinator".

Example:
\begin{figure}[H]
\centering
\begin{circuitikz} \draw
    (0,2) node[nor port] (R) {}
    (1,2) node[anchor=east] {}
    (1,0) node[anchor=east] {}
    (0,0) node[nor port] (S) {}
    (R.out) to (1.5,2)
    (S.out) to (1.5,0)
    (R.in 1) node[anchor=east] {R}
    (R.in 2) node[anchor=east] {X}
    (S.in 1) node[anchor=east] {Y}
    (S.in 2) node[anchor=east] {S}
    (R.out) -- ++(0,-0.5) -- ($(S.in 1) +(0,0.5)$) -- (S.in 1)
    (S.out) -- ++(0,+0.5) -- ($(R.in 2) +(0,-0.5)$) -- (R.in 2)
\end{circuitikz}
\caption{RS Latch}
\end{figure}
\begin{mathpar}
  RSL(R, S) = \fix x {\nor 1 R{\nor 2 x S}}
\end{mathpar}

Attempting to evaluate $RSL(\ff, \ttt)$:
\begin{align*}
  RSL(\ff, \ttt) &= \fix x {\nor 1 \ff {\nor 2 x \ttt}} \\
  &= \nor 1 \ff {\nor 2 {RSL(\ff, \ttt)} \ttt} \\
  &= \nor 1 \ff \ff \\
  &= \ttt
\end{align*}

$RSL(\ff, \ff)$ has no value as it diverges.

\section{Check Coherence}

We may have some undefined terms: $\fix x x$, $\frac{1}{0}$...

What is the meaning of well defined?

Structural properties state that we may substitute computations (v.s. values) for
variables. It is obvious that $e$'s need not evaluate to $v$'s. With \texttt{fix},
coherence can no longer be expressed as evaluating values. So what do we say instead?
The issue is, in general there are many sorts of undefinedness. To handle this, the idea is to
distinguish divergence from nonsense.

Idea: To introduce ``time'' (though it introduces complexity):
\begin{enumerate}
\item Using time to express coherence;
\item Using time to express ``propagation delay''.
\end{enumerate}

$e\ \text{val}$ ``final state''

\begin{mathpar}
  \inferrule*
  { }
  {\ttt\ \text{val}}

  \inferrule*
  { }
  {\ff\ \text{val}}
\end{mathpar}

\textbf{Structural operational semantics(SOS)}

$e \mapsto e'$ denotes a transition system.

\begin{mathpar}
  \inferrule*
  {e_1 \mapsto e'_1}
  {\nor 1{e_1}{e_2} \mapsto \nor 1 {e'_1}{e_2}}

  \inferrule*
  { }
  {\nor 1{\ttt}{e_2} \mapsto \ff}

  \inferrule*
  {e_2 \mapsto e'_2}
  {\nor 1{\ff}{e_2} \mapsto \nor 1{\ff}{e'_2}}

  \inferrule*
  { }
  {\nor 1 \ff \ttt \mapsto \ff}

  \inferrule*
  { }
  {\nor 1 \ff \ff \mapsto \ttt}

  \inferrule*
  {e_2 \mapsto e'_2}
  {\nor 2{e_1}{e_2} \mapsto \nor 1 {e_1}{e'_2}}

  \inferrule*
  { }
  {\nor 2{e_1}{\ttt} \mapsto \ff}

  \inferrule*
  {e_1 \mapsto e'_1}
  {\nor 2{e_1}{\ff} \mapsto \nor 1{e'_1}{\ff}}

  \inferrule*
  { }
  {\nor 2 \ttt \ff \mapsto \ff}

  \inferrule*
  { }
  {\nor 2 \ff \ff \mapsto \ttt}

\end{mathpar}

\begin{enumerate}
\item[\textbf{Preservation}] If $\Circ e$ and $e \mapsto e'$, then $\Circ{e'}$.
\item[\textbf{Progression}] If $\Circ e$ then $e\ \text{val}$ or exists $e'$, $e \mapsto e'$.
\end{enumerate}

Progress-and-preservation denotes safety or coherence. It allows no ill-defined situations yet divergence is OK.

\section{Exercises}

\hypertarget{Exercises}{Exercises:}.

\begin{itemize}
    \item Give an alternate formulation of the circuit language using BDD's (binary decision diagrams).  The abstract syntax for this language is given by,
    \begin{align*}
      e ::= \; x\; |\; \ttt{}\;|\;\ff{}\;|\; \texttt{if}(e;e;e)
    \end{align*}

    \item Prove that $\Eval{e}{v}$ if and only if $\multival{e}{v}$,
    where $\mapsto^{*}$ is a multi-step transition.  We have the additional rules,
    \begin{mathpar}
      \inferrule*
      { }
      {\multival{v}{v}}

      \inferrule
      {e \mapsto e^\prime \\ \multival{e^\prime}{v}}
      {\multival{e}{v}}
    \end{mathpar}

    \item Using structural operational semantics, calculate the number of steps it takes for $RSL(\ff{},\ttt{})$ to transition to $\ttt{}$

\end{itemize}

\end{document}
