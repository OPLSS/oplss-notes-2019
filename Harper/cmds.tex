% Grammar
\newcommand{\arr}[2]{#1 \rightarrow #2}
\newcommand{\arrc}[2]{\mathtt{arr}\left(#1,#2\right)}

\newcommand{\unit}{\mathbf{1}}
\newcommand{\unitc}{\mathtt{unit}}

\newcommand{\app}[2]{#1\left(#2\right)}
\newcommand{\appc}[2]{\mathtt{app}\left(#1,#2\right)}

\newcommand{\lam}[3]{\lambda\left(#1:#2\right).~#3}
\newcommand{\lamc}[3]{\mathtt{lam}\{#2\}\left(#1.#3\right)}

\newcommand{\trivc}{\mathtt{triv}}
\newcommand{\triv}{\langle\rangle}

\newcommand{\fixp}[3]{\mathtt{fix}\left(#1:#2\right).~#3}
\newcommand{\fixpc}[3]{\mathtt{fix}\{#2\}\left(#1.#3\right)}

\newcommand{\tick}[2]{\mathtt{tick}~#1~\mathtt{in}~#2}
\newcommand{\tickc}[2]{\mathtt{tick}\{#1\}\left(#2\right)}

% Types
\newcommand{\isval}[1]{#1\,\mathrm{val}}

\newcommand{\costof}[2]{#1\left(\mathtt{#2}\right)}
\newcommand{\eval}[3]{ #1 \Downarrow^{#2} #3}
\newcommand{\evalenv}[5]{#1 \vdash_{#2} #3 \Downarrow^{#4}#5}
\newcommand{\evalwatermark}[6]{#1 {}_{#2}\!\vdash^{#3}_{#4} #5 \Downarrow #6 }
\newcommand{\evalmono}[6]{#1 \vdash_{#2} #3 \Downarrow #4 \mid \left(#5,#6\right)}

\newcommand{\sharingsymbol}{\curlyvee} % replace this by the correct symbol later
\newcommand{\sharing}[3]{#1\curlyvee#2\mid#3}

\newcommand{\typ}[5]{#1\vdash^{#2}_{#3} #4:#5}
\newcommand{\matchc}[5]{\mathtt{matL}\left(#1,#2,#3.#4.#5\right)}


\usepackage[english]{babel}
 
\usepackage{amsthm}
 
\theoremstyle{definition}
\newtheorem{definition}{Definition}[section]
\newtheorem{remark}{Remark}[section]
\newtheorem*{remark*}{Remark}


\newcommand{\share}[4]{\mathtt{share}~#1~\mathtt{as}~#2,#3~\mathtt{in}~#4}

\usepackage{listings}
\usepackage{color}

\definecolor{dkgreen}{rgb}{0,0.6,0}
\definecolor{gray}{rgb}{0.5,0.5,0.5}
\definecolor{mauve}{rgb}{0.58,0,0.82}

\lstset{frame=tb,
  language=SML,
  aboveskip=3mm,
  belowskip=3mm,
  showstringspaces=false,
  columns=flexible,
  basicstyle={\small\ttfamily},
  numbers=none,
  numberstyle=\tiny\color{gray},
  keywordstyle=\color{blue},
  commentstyle=\color{dkgreen},
  stringstyle=\color{mauve},
  breaklines=true,
  breakatwhitespace=true,
  tabsize=3
}

% Commands for Bob Harper's lecture

\newcommand{\clsfd}{\underline{\mathtt{clsfd}}}
\newcommand{\inharp}[2]{\underline{\mathtt{in}}\left[#1\right](#2)}
\newcommand{\isin}[4]{\underline{\mathtt{isin}}\left[#1\right](#2,#3,#4)}
\newcommand{\new}[2]{\underline{\mathtt{new}}\left[#1\right](#2)}
\newcommand{\ret}[1]{\underline{\mathtt{ret}}\left(#1\right)}
\newcommand{\clsty}{\underline{\mathtt{cls}}}
\newcommand{\cls}[1]{\underline{\mathtt{cls}}\left[#1\right]}
\newcommand{\reveal}[2]{\underline{\mathtt{reveal}}(#1; #2)}
\newcommand{\inrefharp}[2]{\underline{\mathtt{inref}}(#1; #2)}
\newcommand{\isinref}[4]{\underline{\mathtt{isinref}}(#1; #2; #3; #4)}
\newcommand{\mk}{\mathtt{mk}}
\newcommand{\option}{\underline{\mathtt{opt}}}
\newcommand{\bnd}[3]{\underline{\mathtt{bnd}}(#1, #2, #3)}
\newcommand{\cmp}[1]{\underline{\mathtt{comp}(#1)}}

\newcommand{\approxty}{%
  \mathbin{\text{%
    \mathsurround=0pt
    \ooalign{%
      \hidewidth\vphantom{$\div$}\raisebox{.95ex}{\scalebox{1.2}{.}}\hidewidth\cr
      $\sim$\cr
      \hidewidth\vphantom{$\div$}\raisebox{-.05ex}{\scalebox{1.2}{.}}\hidewidth\cr
    }%
  }}%
}