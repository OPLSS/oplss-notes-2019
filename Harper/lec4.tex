\documentclass[11pt]{article}
\usepackage[margin=1in]{geometry}
\usepackage{amsmath}
\usepackage{amssymb}
\usepackage{amsthm}
\usepackage{mathpartir}
\usepackage{syntax}
\usepackage{listings}
\usepackage{hyperref}
\usepackage{cite}
\usepackage{accents}
\usepackage{tikz-cd}

\newtheorem{thm}{Theorem}[section]
\newtheorem{lemma}[thm]{Lemma}
\newtheorem{corr}{Corollary}
\newtheorem{definition}{Definition}[section]
\newtheorem*{remark}{Remark}

\newcommand{\entails}{\ensuremath{\vdash}}
\newcommand{\sequent}[3]{\ensuremath{#1 \vdash{} #2 \colon{} #3}}
\newcommand{\seqO}[2][x]{\ensuremath{#1_0,#1_1,\ldots,#1_{#2}}}
\newcommand{\seqS}[2][x]{\ensuremath{#1_1,#1_2,\ldots,#1_{#2}}}
\newcommand{\circuit}{\texttt{circ}}
\newcommand{\RR}{\ensuremath{\mathbb{R}}}
\newcommand{\NN}{\ensuremath{\mathbb{N}}}
\newcommand{\ZZ}{\ensuremath{\mathbb{ZZ}}}
\newcommand{\pcf}{\textsc{Pcf}}
\newcommand{\dsym}{\ensuremath{\;\underaccent{\dot}{\dot{\sim}}\;}}

\newcommand{\T}[1]{\ensuremath{\mathtt{#1}}}
\newcommand{\Ltt}{\ensuremath{\mathtt{tt}}}
\newcommand{\Lff}{\ensuremath{\mathtt{ff}}}
\newcommand{\Lcomp}[1]{\ensuremath{#1\;\mathtt{comp}}}
\newcommand{\Lval}{\ensuremath{\;\mathtt{val}}}
\newcommand{\Lzero}{\ensuremath{\mathtt{zero}}}
\newcommand{\Lif}[3]{\ensuremath{\mathtt{if}(#1 ; #2 ; #3)}}
\newcommand{\Lifz}[3]{\ensuremath{\mathtt{ifz}(#1 ; #2 ; #3)}}
\newcommand{\Llam}[3][\tau]{\ensuremath{\lambda[#1](#2 . #3)}}
\newcommand{\Lfix}[3][\tau]{\ensuremath{\mathtt{fix}[#1](#2 . #3)}}
\newcommand{\Lapp}[2]{\ensuremath{#1\, #2}}
\newcommand{\Lbind}{\ensuremath{\mathtt{bind}}}
\newcommand{\Lraise}{\ensuremath{\mathtt{raise}}}
\newcommand{\Lunit}{\ensuremath{\langle\rangle}}

\setlength{\parindent}{0pt}

\lstset{
  basicstyle=\footnotesize\ttfamily,
  breaklines=true,
  captionpos=b,
  keepspaces=true,
  escapeinside={(*@}{@*)},
  showtabs=false
}

\title{Practical Foundations for Programming Languages}
\author{Jason Hu, Zhe Zhou, Ramana Nagasamudram}
\date{}

\begin{document}
\maketitle

\section{Dynamic Classification}
\label{sec:dc}

% (Terminology in PFPL ; synthesis of ideas that have been around for
% decades.)

% Versus \emph{Static classification} = (finite) sums

% $$\mathcal{U}\approxeq [\T{num}\hookrightarrow \omega,\T{nil}\hookrightarrow
% 1,\T{cons}\hookrightarrow \mathcal{U} \times \mathcal{U}, \T{fun}\hookrightarrow \mathcal{U}\rightarrow
% \mathcal{U} \dots]$$

Each summand is a \underline{class} of values (sometimes called run-time
types, but this makes it confusing). \emph{Dynamic dispatch} is just a
case analysis for a sum type.  There is a discussion in PFPL about dynamic
dispatch (to dispel certain assumptions ; discussion of a \emph{dispatch
matrix}).
When you do case analysis of some element of a sum type, you recover information
because you know what the underlying data is.  This is an important idea
-- case analysis reveals information.

\vspace{1em}

\textbf{What we want}: Some type of ``extensible'' sum.  That is a sum of a bunch
of things, e.g. $\tau_{1}+\tau_{2}+\dots$ (we want the $\dots$ to be part
of the syntax).  We want dynamical extensibility.  At run-time, we are
generating new classes of data that we can dispatch on.  We have things
that are dynamically new (as opposed to statically new).  This
means that it will be completely tied with issues of information flow or
knowledge flow.

A dynamically new class corresponds to an ``unguessable'' secret (or
perfect encryption by $\alpha$-conversion).  ``Encryption is just
$\alpha$-conversion and Church invented it in the ``30s''.
It induces \emph{capabilities} express \emph{integrity} and
\emph{confidentiality}.

\textbf{Key idea}: exceptions (i.e. the values associated with an
exception) are ``shared secrets''.  Raiser ensures the integrity of the
data, and you allow the handler to violate the confidentiality of the
data.

\section{FPC (by-value)  example}
\label{sec:fpc}

We assume that there is a exception control mechanism.

\begin{align*}
  \tau ::=& \dots \mid \T{clsfd} \mid \tau\;\T{cls} \\
  v ::=& \dots \mid \T{in}[c](v) \mid \&c \\
  m ::=& \dots \mid \T{isin}[c](v;x.m_{1};m_{2}) 
  \mid \T{new}[\tau](c.m) \mid \T{reveal}(v ; c.m)
\end{align*}

$\T{clsfd}$ is dynamically classified -- we are going to attach a class to
a data item, and we can generate these classes at runtime.  The classes
$c$ are an indexing scheme (that happens to be
open-ended). $\tau\;\T{cls}$ is a ``class reference''.  (The book contains
a section on references that has nothing to do with state).  $\T{cls}[c]$
is written as $\& c$.

\begin{mathpar}
  \inferrule* [Right=]
  {}
  {\Gamma{}\entails_{\Sigma} v : \tau}

  \inferrule* [Right=]
  {}
  {\Gamma{}\entails_{\Sigma{}} m \dsym \tau}
\end{mathpar}

$\Sigma = c_{1} \sim \tau_{1}, \dots, c_{n} \sim \tau_{n}$ are
``signatures''. Order doesn't matter. It can be weakened, do \emph{injective}
renaming (because we need the stability of disequality).  These are not
variables.


$\T{new}[\tau](c.m)$ is a class binder (binds the class $c$ in $m$)

\subsection{Statics}

\begin{mathpar}
  \inferrule*
  {\Gamma \vdash_{\Sigma, c \sim \tau} v : \tau}
  {\Gamma \vdash_\Sigma \T{in}[c](v) : \T{clsfd}}

  \inferrule*
  {\Gamma \vdash_{\Sigma, c \sim \tau} v : \T{clsfd} \\
  \Gamma,x : \tau \vdash_{\Sigma, c \sim \tau} m_1 \dsym \tau \\ \Gamma \vdash_{\Sigma, c \sim \tau} m_2 \dsym \tau}
  {\Gamma \vdash_{\Sigma, c \sim \tau} \T{isin}[c](v;x.m_{1};m_{2}) \dsym \tau}

  \inferrule* [Right=]
  {\Gamma{}\entails_{\Sigma,c\sim\tau} m \dsym \tau^{\prime}}
  {\Gamma{}\entails_{\Sigma}\T{new}[\tau](c.m) \dsym \tau^{\prime}}

  \inferrule* [Right=]
  { }
  {\Gamma \entails_{\Sigma,c\sim\tau} \& c : \tau\;\T{cls}}

  \inferrule* [Right=]
  {\Gamma\entails_{\Sigma} v : \tau\;\T{cls} \\
    \Gamma\entails_{\Sigma,c\sim\tau} m \dsym\tau^{\prime}}
  {\Gamma \entails_{\Sigma}\T{reveal}(v;c.m)\dsym\tau^{\prime}}
\end{mathpar}

\subsection{Dynamics}

In a (scope-)free style,

states : $\nu\Sigma\{m\}$

transitions : $\nu\Sigma\{m\}\mapsto \nu\Sigma^{\prime}\{m^{\prime}\}$
with the invariant of $\Sigma^{\prime} \supseteq \Sigma$

final states : $\nu\Sigma\{\T{ret}(v)\}$

When is a state statically OK? $\nu\Sigma\{m\}\;\T{ok}$ means
$\entails_{\Sigma} m \dsym\tau$(for any $\tau$). Compare with what we do for the static type($m \mapsto_{\Sigma} m^{\prime}$).

\begin{mathpar}
  \inferrule* [Right=]
  { }
  { \nu\Sigma\{\T{new}[\tau](c.m)\}\mapsto \nu(\Sigma , c\sim\tau)\{m \} }

  \inferrule* [Right=]
  { c = c^\prime (\Rightarrow \tau = \tau^\prime)}
  {\nu(\Sigma,c\sim\tau,c^{\prime}\sim\tau^{\prime})\{
    \T{isin}[c](\T{in}[c^{\prime}](v^{\prime}) ; x . m_1 ; m_2 )\}
    \mapsto\nu(\Sigma,c\sim\tau,c^{\prime}\sim\tau^{\prime})\{[v^{\prime}/x]m_1\}}

  \inferrule* [Right=]
  { c \neq c^\prime }
  {\nu(\Sigma,c\sim\tau,c^{\prime}\sim\tau^{\prime})\{
    \T{isin}[c](\T{in}[c^{\prime}](v^{\prime}) ; x . m_1 ; m_2 )\}
    \mapsto \nu(\Sigma,c\sim\tau,c^{\prime}\sim\tau^{\prime})\{m_2\}}

  \inferrule* [Right=]
  { }
  {\nu(\Sigma,c\sim\tau^{\prime})\{\T{reveal}(\& c; c.m)\}\mapsto
    \nu(\Sigma,c\sim\tau)\{ m \}}
\end{mathpar}

$$\T{inref}(v,v^{\prime})\triangleq \T{reveal}(v;c.m[c](v^{\prime})) \dsym
\T{clsfd}$$

With all of this you can form a classified value without actually knowing
what the classified value is.

\textbf{Exercise}: define $\T{isinref}(v ; v^{\prime} ; x.m_{1} ; m_{2})$ (like
$\T{isin}$ but dynamically determined).

\section{Capabilities}

Possession of a reference (class reference) affords ``full rights'' to
create and analyze classified values (according to that reference).
$$\tau\;\T{cls}\approx\langle \T{mk}\hookrightarrow\tau\rightarrow\T{clsfd},
\T{mk}^{-1}\hookrightarrow \T{clsfd}\rightarrow\tau\;\T{opt}\rangle$$, where $\tau\;\T{opt}$ is just $\tau + 1$.

$$\& a \mapsto\langle \T{mk}\hookrightarrow \lambda[\tau](x.\T{in}[a](x)),
\T{mk}^{-1}\hookrightarrow
\lambda[\T{clsfd}](x.\T{isin}[a](x;y.\T{ret}(1\cdot y);
\T{ret}(2\cdot\Lunit))\rangle$$

What we want is dynamically "extensible" sum("$\tau_1 + \tau_2 + ... + \tau_n$"), and dynamically "new", aka, information flow or knowledge flow.

$\texttt{class}$ is unguessable secret or perfect encryption(by $\alpha-$conversion).

\underline{capabilities} express "integrity" and "confidentiality".

exceptions (i.e. the values) are "shared secrets".

\begin{enumerate}
\item The possessor of $\T{mk}$ is held responsible for the
\underline{integrity} of the underlying data.  The possessor, therefore, must ensure that invariant about the data hold.
\item The possessor of $\T{mk}^{-1}$ can compromise the \underline{confidentiality} barrier and know that the associated invariant holds.
\end{enumerate}

\begin{tikzcd}
                            &  & mk      \\
\&a \arrow[rru] \arrow[rrd] &  &         \\
                            &  & mk^{-1}
\end{tikzcd}

(To do some enforcement or analysis of security properties seems to involve
some form of epistemic logic)

\vspace{1em}

\subsection{Relation to ML}

In ML,
\begin{enumerate}
\item $\T{exception}\ X \;\T{of}\; \tau$ declares an exception $X$ carrying
values of type $\tau$.
\item $\T{raise}(e)$ raises an exception $e : \T{exn}$ where $\T{exn}$ is just
$\T{clsfd}$.
\item $e\;\T{handle}\; z\Rightarrow e$ handles an exception: You don't
need the name of the exception to handle it!
\end{enumerate}

\begin{lstlisting}
  exception X of (*@$\tau'$@*)
  val X = (*@$\langle$@*)raise[(*@$\tau$@*)] (*@$\hookrightarrow$@*) exn (*@
  $\rightharpoonup \tau$ @*),
           handle[(*@$\tau$@*)] (*@$\hookrightarrow$@*) (*@$\tau \rightharpoonup
(\T{clsfd}\rightharpoonup\tau)\rightharpoonup\tau$@*)(*@$\rangle$@*)

  raise X(v) (*@$\hookrightarrow$@*) bnd(comp(x.mk(v));
                    y.RAISE(y);
                    z.RAISE(t))

  e handle X(x) => m' (*@$\hookrightarrow$@*)
    bnd(comp(m); x. ret(x);
        y. bnd(comp(x.mk(*@$^{-1}$@*)(y)); x.m'; z.RAISE(z)))

  exception X of (*@$\tau$@*) (*@$\hookrightarrow$@*)
    new[(*@$\tau$@*)](c.(*@$\langle$@*)mk (*@$\hookrightarrow\dots$@*)
              mk(*@$^{-1}$@*) (*@$\hookrightarrow\dots$@*)(*@$\rangle$@*))
\end{lstlisting}

In $\T{raise}\; X(v)$, $X$ is of type $\tau\rightarrow\T{clsfd}$ and $v$
is of type $\tau$.

\section{Things to look at}
\label{sec:look}

\begin{itemize}
\item Datatypes are abstract (\emph{statically new}) recursive sums (static
  classification).  Even if you write the same datatype declaration twice,
  you get two different types.  Sometimes people call this
  a \emph{generative data type}.
\end{itemize}

\section{References}

\begin{itemize}
\item Robert Harper. PFPL Supplement: Exceptions: Control and Data \url{https://www.cs.cmu.edu/~rwh/pfpl/supplements/exceptions.pdf}
\item Robert Harper. Practical Foundations for Programming Languages. Chapter 29, 33.
\end{itemize}

\end{document}
